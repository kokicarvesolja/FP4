\documentclass[12pt]{report}

\usepackage[a4paper]{geometry}
%\geometry{left=2.5cm,right=2.5cm,top=2.5cm,bottom=2.5cm, a4paper}
\usepackage[utf8]{inputenc}
\usepackage{amsmath}
\usepackage{amsthm}
\usepackage{amssymb}
\usepackage{ulem}
\usepackage{graphicx}
\usepackage{caption}
\graphicspath{}
\usepackage[document]{ragged2e}
\usepackage{setspace}
\usepackage{tabularx}
\usepackage[slovene]{babel}
\usepackage{textcomp, gensymb}
\usepackage{siunitx}
\usepackage{pdfrender,xcolor}
\usepackage{hyperref}
\usepackage{xurl}
\usepackage{float}
\usepackage{titlesec}

\newfloat{slika}{htbp}{loc}
\floatname{slika}{Slika}

\newfloat{tabela}{htbp}{loc}
\floatname{tabela}{Tabela}

% Differential
\newcommand{\diff}{\mathrm{d}}

\title{
  \includegraphics[width=0.4\textwidth]{fmf_logo}\\
  {\small Oddelek za fiziko} \\
  {Sklopljena nihajna kroga}\\
  {\small Poročilo pri fizikalnem praktikumu IV}\\

}
\date{}
\author{ Kristofer Č. Povšič \\[5 cm]
 \small  Asistentka: Jelena Vesić
}


\titleformat{\chapter}[hang]{\Huge\bfseries}{\thechapter{. }}{0pt}{\Huge\bfseries}

\setlength\parindent{0pt}

\begin{document}

\setcounter{page}{2}

\maketitle

\chapter*{Uvod}

Pri sklopitvi povzročimo, da posameznih oscilatorjev več ne obravnavamo ločeno, ampak kot en sistem. Sistem sestavljen iz $n$ enakih oscilatorjev, ima $n$ lastnih nihanj, ki jih opišemo z lastnimi frekvencami $\omega_n$ in lastnimi vektorji. 

Ko povežemo dva identična nihajna kroga s kondenzatorjem $C_0$, je en način nihanja, da nihata v fazi in vmesnega sklopitvenega kondenzatorja ne zaznata. Drugi način pa je, da nihata v nasprotni fazi. 

Rešitev diferencialne enačbe za zgolj kapacitivno sklopljena kroga nam za začetni pogoj, kjer drugi krog miruje in začnemu vzbujati prvi krog, napove odvisnost napetosti oblike

\[
  \begin{align*}
    U_1 = U_0 e^{-\beta t}\cos(\omega t) \cos (\Delta \omega t)
    U_2 = U_0 e^{-\beta t} \cos(\omega t) \sin (\Delta \omega t)
  \end{align*}
\]

Naša eksperimentalna postavitev pa ni ravno tako idealna - predvsem je problem induktivna sklopitev, ki pri manjših vrednostih nastavljivega kondenzatorja igra precej veliko vlogo, a smo jo tu ignorirali. Ko merimo resonančno krivuljo, vidio, da je resonačni vrh položnejši, čim večje je dušenje $\beta$. Pogosto namesto parametra $\beta$ navajamo doboroto oz. kvaliteto nihajnega kroga 

\begin{equation}
  Q = \frac{\omega_1}{\Delta \omega} = \frac{omega}{2 \beta} = \sqrt{\frac{L}{CR^2}}
\end{equation}

pri čemer je $\omega_1$ resonančna frekvenca, $\Delta \omega$ širina resonančne krivulje pri $\frac{1}{\sqrt{2}}$ maksimuma. 

\chapter*{Naloga}

\begin{itemize}
  \item Izmerite časovni potek napetosti na obeh krogih pri vzbujanju s stopničastim signalom za vse različne sklopitve $C_0 = 0,\, 150,\, 330,\, 560,\, 820,\, 1150 \si{pF}$. 
  \item Izmerite frekvenčno karakteristiko enega nihajnega kroga in določite dobroto $Q$. 
  \item Izmerite frekvenčno karakteristiko sklopljenih nihajnih krogov z meritvijo odziva drugega kroga za vsak $C_0$ in izmerite razliko lastnih krožnih frekvenc $\Delta \omega$. 
\end{itemize}


\begingroup
\let\clearpage\relax

\chapter*{Potrebščine}
\begin{itemize}
  \item digiatlni osciloskop
  \item funkcijski generator napetosti, namizni multimeter
  \item nihajna kroga in kabli, USB ključek
  \item prenosnik s programom \verb+SkNikKr+ napisan v LabView
\end{itemize}

\chapter*{Navodilo}

\begin{enumerate}
  \item Odziv obeh nihajnih krogov na napetostno stopnico: Povežem stvari kot je v navodilih in potem spreminjam sklopitveni kondenzator in opazujem signale v obeh krogih. 
  \item Vsiljeno nihanje enega nihajnega kroga: Sklopitveni kondenzator $C_0$ naj bo izklopljen, torej $C_0 = 0$ in kratko sklenimo drugi nihajni krog, tako da povežemo izhod $U_2$ in zemljo. Nato odstrani še kratko sklenitev kroga. 
  \item Vsiljeno nihanje sklopljenih krogov: Pri vklopljenem sklopitvenem kondenzatorju lahko opazimo resonančno obnašanje na obeh krogih, vendar je na drugem krogu bolj izrazito. S programom izmerite frekvenčno odvisnost od efektivne napetosti na drugem krogu v istem frekvenčnem intervalu kot pri 1. nalogi. 

\endgroup


\chapter*{Obdelava podatkov}

\section*{1. del}





\end{document}