\documentclass[12pt]{report}

\usepackage[a4paper]{geometry}
%\geometry{left=2.5cm,right=2.5cm,top=2.5cm,bottom=2.5cm, a4paper}
\usepackage[utf8]{inputenc}
\usepackage{amsmath}
\usepackage{amsthm}
\usepackage{amssymb}
\usepackage{ulem}
\usepackage{graphicx}
\usepackage{caption}
\graphicspath{}
\usepackage[document]{ragged2e}
\usepackage{setspace}
\usepackage{tabularx}
\usepackage[slovene]{babel}
\usepackage{textcomp, gensymb}
\usepackage{siunitx}
\usepackage{pdfrender,xcolor}
\usepackage{hyperref}
\usepackage{xurl}
\usepackage{float}
\usepackage{titlesec}

\newfloat{slika}{htbp}{loc}
\floatname{slika}{Slika}

\newfloat{tabela}{htbp}{loc}
\floatname{tabela}{Tabela}

% Differential
\newcommand{\diff}{\mathrm{d}}

\title{
  \includegraphics[width=0.4\textwidth]{fmf_logo}\\
  {\small Oddelek za fiziko} \\
  {Vrtavka}\\
  {\small Poročilo pri fizikalnem praktikumu IV}\\

}
\date{}
\author{ Kristofer Č. Povšič \\[5 cm]
 \small Asistentka: Jelena Vesić \\
}


\titleformat{\chapter}[hang]{\Huge\bfseries}{\thechapter{. }}{0pt}{\Huge\bfseries}

\setlength\parindent{0pt}

\begin{document}

\setcounter{page}{2}

\maketitle

\chapter*{Uvod}

Opisujemo gibanje vrtavke, kjer je njena kinetična energija mnogo večja od potencialne. Vrtavka ima dolgo os poravnano z $z$-osjo, je nutacijska frekvenca podana z 

\begin{equation}
  \nu_{nu} = \frac{J_{33}}{J_{11}}\nu_z
\end{equation}

kjer ima vektor vztrajnostnega momenta komponente $\vec{J} = (J_{11}, J_{22}, J_{33})$ ter je $\nu_z$ frekvenca kroženja. Precesijska frekvenca je podana z

\begin{equation}
  \nu_{pr} = \frac{1}{4\pi^2}\frac{mgh^*}{J_{33}\nu_z}
\end{equation}

kjer je $m$ masa vrtavke, $g$ težnostni pospešek, $h^*$ pa razdalja med težiščem in oporno točko osi ($h^*$ je ročica navora). 

\chapter*{Naloga}

Izmeri precesijsko ($\omega_{pr}$) in nutacijsko kotno hitrost ($\omega_{nu}$) v odvisnosti od kotne hitrosti ($\omega_z$) vrtavke. Izvedi meritev pri vsaj treh frekvencah $\nu_z = \frac{\omega_z}{2\pi}$. Na primer pri približno 600, 500, 400 obratih na minuto (kratica rpm - angl. rotations per minute). Gornjo meritev izvedi pri naslednjih nastavitvah vrtavke: 
\begin{itemize}
  \item vrtavka z utežjo blizu krogle
  \item utež na sredini palice 
  \item utež na koncu palice (pusti si prostor za oprijem)
\end{itemize}
Meritve z različnimi nastavitvami vrtavke izvedi pri podobnih frekvencah $\nu_z$ kot prej, da so rezultati lažje primerljivi. Izmerjene vrednosti $\omega_{pr}$ in $\omega_{nu}$ primerjaj z izračunanimi in naredi tabelo. 

\begingroup
\let\clearpage\relax

\chapter*{Potrebščine}
\begin{itemize}
\item krogla s podnožjem in priborom (palica, utež in ploščica z vzorcem)
\item kompresor pod mizo 
\item stroboskop
\item štoparica 
\end{itemize}


\chapter*{Navodilo}

Sestavim vajo po navodilih. Vrtavka je sestavljen aiz kovinske krogle, aluminijaste palice s snemljivivo in premakljivo utežjo. Na palici je fiksno pritrjena ploščica z močno kontrastnim vzorcem. Pod stroboskopom lahko s pomočjo vzorca določim frekvenco rotacije okoli lastne osi vrtavke $\nu_z$. 

Vključim kompresor in ventilator. S prsti zavrtim vrtavko v navpičnem položaju in pustim, da jo zračni curek dodatni zavrti. Spreminjam frekvenco na stroboskopu in ko se vzorec na vrtavki ustali, odčitam izmerjeno vrednost vrtenja. Prislonim svinčnik in s tem izmaknem vrtavko iz osi in vrtavka začne s precesijo. Izmerim precesijski čas pod različinimi koti, da vidim, da je neodvisen od kota. 

S kratkim udarcem po vrtavki, začnem nutacijo. Potrebno jo je večkrat vzbuditi, saj jo trenje zavira. 

Znani podatki o vrtavki so: 

\begin{equation*}
  m_s = \SI{515}{g} \qquad 2r_s = \SI{50.8}{mm},
\end{equation*}

in

\begin{align*}
  m_r &= \SI{15}{g} \qquad 2r_r = \SI{51}{g} \qquad h_r = \SI{1.1}{mm} \\
  m_b &= \SI{27}{g} \qquad 2r_b = \SI{6.5}{g} \qquad h_b = \SI{100.5}{mm} \\
  m_w &= \SI{18}{g} \qquad 2r_1 = \SI{20}{g} \qquad h_w = \SI{25.2}{mm},
\end{align*}

\endgroup


\chapter*{Obdelava podatkov}

\section*{Utež na $h^*=0$}

Izmerjeni podatki so: 

\begin{tabela}[H]
  \centering
  \[
  \begin{array}{|cc|cc|c|}\hline
    (\nu_z \pm 50) [\si{rpm}] & (\nu_z \pm 0.8) [\si{Hz}] & 5t_{pr} [\si{s}] & \nu_{pr} [\si{Hz}] & \nu_{nu} [\si{Hz}] \\ \hline 
    350 & 5.8 & 15.8 & 0.3 \pm 0.1 & 3.7 \pm 0.1 \\
    510 & 8.5 & 15.7 & 0.3 \pm 0.1 & 4.6 \pm 0.1 \\
    600 & 10 & 18.0 & 0.2 \pm 0.1 & 8.4 \pm 0.1 \\ \hline 
  \end{array}
  \]
\end{tabela}

Izračunani podatki so: 

\begin{tabela}[H]
  \centering
  \[
  \begin{array}{|c c|c c|c c|}\hline 
    \nu_z [\si{rpm}] & \pm [\si{rpm}] & \nu_{pr} [\si{Hz}] & \pm [\si{Hz}] & \nu_{nu} [\si{Hz}] & \pm [\si{Hz}] \\\hline
    350.00 &   50.00 &    0.79 &    0.12 &    2.86 &    0.41\\
    510.00 &   50.00 &    0.55 &    0.06 &    4.17 &    0.41\\
    600.00 &   50.00 &    0.46 &    0.04 &    4.90 &    0.41 \\ \hline 
  \end{array}
  \]
\end{tabela}

\section*{Utež na $h^*=(3.5 \pm 0.1) \si{cm}$}

Izmerjeni podatki so: 

\begin{tabela}[H]
  \centering
  \[
  \begin{array}{|c c|c c| c|}\hline
    (\nu_z \pm 50) [\si{rpm}] & (\nu_z \pm 0.8) [\si{Hz}] & 5t_{pr} [\si{s}] & \nu_{pr} [\si{Hz}] & \nu_{nu} [\si{Hz}] \\ \hline 
    360 & 6 & 9.0 & 0.5 \pm 0.1 & 2.3 \pm 0.1 \\
    510 & 8.5 & 9.6 & 0.5 \pm 0.1 & 3.1 \pm 0.1 \\
    580 & 9.7 & 11.2 & 0.4 \pm 0.1 & 3.6 \pm 0.1 \\ \hline 
  \end{array}
  \]
\end{tabela}

Izračunani podatki so: 

\begin{tabela}[H]
  \centering
  \[
  \begin{array}{|c c|c c|c c|}\hline
    \nu_z [\si{rpm}] & \pm [\si{rpm}] & \nu_{pr} [\si{Hz}] & \pm [\si{Hz}] & \nu_{nu} [\si{Hz}] & \pm [\si{Hz}] \\\hline
    360.00 &   50.00 &    0.91 &    0.13 &    2.60 &    0.35\\
    510.00 &   50.00 &    0.66 &    0.07 &    3.58 &    0.35\\
    580.00 &   50.00 &    0.58 &    0.05 &    4.07 &    0.36\\ \hline 
  \end{array}
  \]
\end{tabela}

\section*{Utež na $h^*=(6.5 \pm 0.1) \si{cm}$}

Izmerjeni podatki so: 

\begin{tabela}[H]
  \centering
  \[
  \begin{array}{|c c|c c| c|}\hline
    (\nu_z \pm 50) [\si{rpm}] & (\nu_z \pm 0.8) [\si{Hz}] & 5t_{pr} [\si{s}] & \nu_{pr} [\si{Hz}] & \nu_{nu} [\si{Hz}] \\ \hline 
    300 & 5 & 3.23 & 1.5 \pm 0.1 & 1.8 \pm 0.1 \\
    410 & 6.8 & 6.8 & 0.7 \pm 0.1 &  3.6\pm 0.1 \\
    510 & 8.5 & 11.3 & 0.4 \pm 0.1 & 4.7 \pm 0.1 \\ \hline 
  \end{array}
  \]
\end{tabela}

Izračunani podatki so: 

\begin{tabela}[H]
  \centering
  \[
  \begin{array}{|c c|c c|c c|}\hline 
    \nu_z [\si{rpm}] & \pm [\si{rpm}] & \nu_{pr} [\si{Hz}] & \pm [\si{Hz}] & \nu_{nu} [\si{Hz}] & \pm [\si{Hz}] \\\hline
    300.00 &   50.00 &    1.29 &    0.22 &    1.74 &    0.29\\
    410.00 &   50.00 &    0.94 &    0.12 &    2.38 &    0.29\\
    510.00 &   50.00 &    0.76 &    0.08 &    2.96 &    0.30\\ \hline 
  \end{array}
  \]
\end{tabela}

\end{document}