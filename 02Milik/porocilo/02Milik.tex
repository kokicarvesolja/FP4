\documentclass[12pt]{report}

\usepackage[a4paper]{geometry}
%\geometry{left=2.5cm,right=2.5cm,top=2.5cm,bottom=2.5cm, a4paper}
\usepackage[utf8]{inputenc}
\usepackage{amsmath}
\usepackage{amsthm}
\usepackage{amssymb}
\usepackage{ulem}
\usepackage{graphicx}
\usepackage{caption}
\graphicspath{}
\usepackage[document]{ragged2e}
\usepackage{setspace}
\usepackage{tabularx}
\usepackage[slovene]{babel}
\usepackage{textcomp, gensymb}
\usepackage{siunitx}
\usepackage{pdfrender,xcolor}
\usepackage{hyperref}
\usepackage{xurl}
\usepackage{float}
\usepackage{titlesec}

\newfloat{slika}{htbp}{loc}
\floatname{slika}{Slika}

\newfloat{tabela}{htbp}{loc}
\floatname{tabela}{Tabela}

% Differential
\newcommand{\diff}{\mathrm{d}}

\title{
  \includegraphics[width=0.4\textwidth]{fmf_logo}\\
  {\small Oddelek za fiziko} \\
  {Določitev osnovnega naboja po Millikanu}\\
  {\small Poročilo pri fizikalnem praktikumu IV}\\

}
\date{}
\author{ Kristofer Č. Povšič \\[5 cm]
 \small Asistent: Jelena Vesić
}


\titleformat{\chapter}[hang]{\Huge\bfseries}{\thechapter{. }}{0pt}{\Huge\bfseries}

\setlength\parindent{0pt}

\begin{document}

\setcounter{page}{2}

\maketitle

\chapter*{Uvod}

Z Millikanovim poskusom opazujemo gibanje naelektrenih kapljic v gravitacijskem in električnem polju. Zaradi relativne enostavnosti je poskus dostopen tudi nam, študentom. 

Poskus izvedemo na dva načina. Ko je vsota sil na kapljico 0 in se premika s konstantno hitrostjo, nanjo delujejo 3 sile; sila teže $mg = \frac{4\pi}{3} r^3 g$, sila vzgona $\frac{4\pi}{3} r^3 \rho_{zr} g$ in Stokesova sila $6\pi r \eta v$, kjer je $\rho_{zr}$ gostota zraka, $\eta$ viskoznostni koeficient, ki je za zrak pri $23 ^\circ C$ enak $18.3 \mu Pas$. Iz teh treh sil lahko izračunamo radij kapljice

\begin{equation}
  r^2 = \frac{9\eta v}{2 (\rho - \rho_{zr})g}
\end{equation}

Ko je kapljica naelektrena in nosi mnogokratni osnovnega naboja $ne_0$, deluje nanjo v električnem polju ploščatega kondenzatorja z električno jakostjo $E$ dodatna sila $ne_0 E$. Dosežemo lahko, da kapljica miruje tako, da spremenimo velikost in smer električnega polja. Takrat velja sledeča enačba

\begin{equation}
  \frac{4\pi}{3} r^3 (\rho - \rho_{zr}) g = ne_0 E 
\end{equation}

kjer je $U = dE$ napetost na kondenzatorju in d je razdalja med ploščama kondenzatorja. Z meritvijo hitrosti pri prostem padanju skozi zrak in napetost, pri kateri se kapljica ustavi, lahko določimo mnogokratnik osnovnega naboja. 

Drugi način pa je, da majhno kapljico premikano z napetostjo $U=dE$ v pozitivni in negativni smeri težnostnega pospeška. Ko se hitrosti ustali velja enakost sil: 

\begin{equation}
  \frac{4\pi}{3} r^3 (\rho - \rho_{zr}) g \pm |n|e_0 E = \pm 6\pi r \eta v_{\pm}
\end{equation}

Hitrosti $v_+$ in $v_-$ sta hitrosti premikanja v pozitivni in negativni smeri težnostnega pospeška in ju lahko izmerimo. Na njihovi podlagi določimo radij kapljice 

\begin{equation}
  r^2 = \frac{9\eta (v_+ - v_-)}{4g(\rho - \rho_{zr})}
\end{equation}

in absolutno vrednost večkratnika naboja $n$: 

\begin{equation}
  |n|e_0 = \frac{3\phi r \eta}{E}(v_+ + v_-)
\end{equation}
  
\chapter*{Naloga}

\begin{itemize}
  \item Izmeri hitrosti gibanja kapljic v gravitacijskem in električnem polju. 
  \item Iz meritev izračunaj velikosti kapljic in njihov naboj ter določi osnovni naboj. 
\end{itemize}


\begingroup
\let\clearpage\relax

\chapter*{Potrebščine}
\begin{itemize}
  \item Millikanov aparat: kondenzator z razmikom $d = 5(1 \pm 0.02)$ mm, razpršilec z oljem ($\rho = 0.973g cm^{-3}$), LED za osvetljevanje 
  \item mikroskop s kamero, ki je priključena na računalnik 
  \item usmernik za 300V
  \item preklopnik smeri napetosti 
  \item voltmeter
\end{itemize}

\chapter*{Navodilo}

Vklopim računalnik in napajalec za belo LEDico, ki osvetljuje notranjost kondenzatorja. Oljne kapljice s stiskom gumijastega balona razpršilca vbrizga, skozi luknjico na zgornji plošči kondenzatorja, ki jih na temnem zaslonu opazim kot svetle točke. Nabite kapljice lahko spuščam gor ali dol s spreminjanjem napetosti preko usmernika za 300V. Posnamem zaslon po prvem in drugem načinu to analiziram preko programa, ki mi potem izračuna hitrost izbrane kapljice, kar si zapišem. 

\endgroup


\chapter*{Obdelava podatkov}

Za račun potrebujem tudi gostoto olja, zraka ter njegovo viskoznost: 
\begin{align*}
  \rho &= (973 \pm 1)\,\si{kg/m^3}, \\
  \rho_{zr} &= (120 \pm 5)\,\si{kg/m^3}, \\
  \eta_{zr} &= (18.3 \pm 0.1)\,\si{\mu Pas}
\end{align*}





\end{document}