\documentclass[12pt]{report}

\usepackage[a4paper]{geometry}
%\geometry{left=2.5cm,right=2.5cm,top=2.5cm,bottom=2.5cm, a4paper}
\usepackage[utf8]{inputenc}
\usepackage{amsmath}
\usepackage{amsthm}
\usepackage{amssymb}
\usepackage{ulem}
\usepackage{graphicx}
\usepackage{caption}
\graphicspath{}
\usepackage[document]{ragged2e}
\usepackage{setspace}
\usepackage{tabularx}
\usepackage[slovene]{babel}
\usepackage{textcomp, gensymb}
\usepackage{siunitx}
\usepackage{pdfrender,xcolor}
\usepackage{hyperref}
\usepackage{xurl}
\usepackage{float}
\usepackage{titlesec}

\newfloat{slika}{htbp}{loc}
\floatname{slika}{Slika}

\newfloat{tabela}{htbp}{loc}
\floatname{tabela}{Tabela}

% Differential
\newcommand{\diff}{\mathrm{d}}

\title{
  \includegraphics[width=0.4\textwidth]{fmf_logo}\\
  {\small Oddelek za fiziko} \\
  {Karakteristika $I(U)$ elektronskih elementov}\\
  {\small Poročilo pri fizikalnem praktikumu IV}\\

}
\date{}
\author{ Kristofer Č. Povšič \\[5 cm]
 \small  Asistentka: Jelena Vesić
}


\titleformat{\chapter}[hang]{\Huge\bfseries}{\thechapter{. }}{0pt}{\Huge\bfseries}

\setlength\parindent{0pt}

\begin{document}

\setcounter{page}{2}

\maketitle

\chapter*{Uvod}

Pri vaji se spoznamo z uporabo funkcijskega generatorja in osciloskopa ter odziv različnih elektronskih elementov. Odziv elementov je lahko linearen in odvisen od frekvence (npr. idealni kondenzator ali tuljava), lahko pa je nelinearen (npr. polprevodniški elementi) ali pa bolj zapleten, če je sestavljen iz različnih sklopov (npr. tuljava z železnim jedrom). 

\chapter*{Naloga}
\begin{enumerate}
  \item Izmerite karakteristiko $I(U)$ upornika, kondenzatorja, tuljave, diode, Zenerjeve diode, treh svetlečih diod, $9\si{V}$ alkalne baterije in akumulatorja
  \item Določite uporanost upornika, kapaciteto kondenzatorja, induktivnost tuljave, karakteristične točke odvisnosti nelinearnih elementov, nazivno napetost in notranjo upornost baterije in akumulatorja. 
\end{enumerate}


\begingroup
\let\clearpage\relax

\chapter*{Potrebščine}
\begin{itemize}
  \item funkcijski generator (GW Instek SFG-2120), ločilni transformator
  \item vezje s komponentami, baterija $9\si{V}$, NiMH akumulator 1.2V, žice
  \item digitalni osciloskp (Siglent SDS 1104X-E)
  \item USB ključek
\end{itemize}

\chapter*{Obdelava podatkov}

\section*{Upornik}

Ima linearno karakteristiko v širokem razponu frekvenc. Tok in napetost sta v fazi in iz naklona se določi R. 

\begin{slika}[H]
  \centering
  \includegraphics{upornik}
  \caption{\small Skica karakteristike upornika pri $\nu = 50\si{Hz}$}
\end{slika}

\[
  R = \frac{\Delta y}{\Delta x R_{not}} = \frac{3.7\si{V}}{3.75\si{V} \cdot 1k\omega} = (1010 \pm 30) \Omega
\]

\section*{Kondenzator}

Preverite frekvenčni razpon, v katerem sta fazi toka in napetosti premaknjeni za $\frac{\pi}{2}$. Določite $C$ iz meritve toka in napetosti pri znani frekvenci. 

\begin{slika}[H]
  \centering
  \includegraphics{kondenzator}
  \caption{\small Skica karakteristike kondenzatorja pri $\nu = 50\si{Hz}$}
\end{slika}

Pri faznem zamiku $\frac{\pi}{2}$ velja: 

\[
 C = \frac{\Delta y}{\Delta x R_{not} \omega} = \frac{1.25\si{V}}{3.5\si{V}\cdot 2\pi \cdot 50\si{Hz} \cdot 1k\omega} = (1.14 \pm 0.04)\mu \si{F}
\]

\section*{Tuljava z jedrom}

Tuljava z jedrom je eden manj idealnih elementov, ker vsebuje dolge žice, ki prispevajo k upornosti in pri višjih frekvenca k parazitski kapacitivnosti ter magnetni odziv feritnega materiala je nelinearen in frekvenčno odvisen. 

Pri faznem zamiku $\frac{\pi}{2}$

\[
  L = \frac{Z_L}{\omega} = \frac{\Delta x R}{\omega \Delta y} = \frac{3.8\si{V} \cdot 1k\Omega}{0.38\si{V} \cdot 2\pi \cdot 10^4\si{Hz}} = (160 \pm 5) mH 
\]

\begin{slika}[H]
  \centering
  \includegraphics{tuljava}
  \caption{\small Skica karakteristike tuljave pri $\nu = 10\si{kHz}$}
\end{slika}

\section*{LED diode}

Nelinearni elementi (diode, LED diode) imajo v karakteristiki kolena. 

Napetosti za tok $|I| = 1\si{mA}$:

\begin{align*}
  U_{bela} &= (2.6 \pm 0.1)\si{V} \\
  U_{IR} &= (1.15 \pm 0.05)\si{V} \\ 
  U_{rdeca} &= (1.75 \pm 0.05)\si{V}
\end{align*}

Vidimo, da je barva LED diode povezana z napetostjo, kjer dioda začne prevajati. IR LED dioda, ki ima večjo valovno dolžino $\lambda$ potrebuje manjši tok, da začne prevajati. Rdeča dioda potrebuje malo večji tok in bela dioda še večjega.

Karakteristika teh elementov se ne spreminja s frekvenco. 

\begin{slika}[H]
  \centering
  \includegraphics{LED}
  \caption{\small Skica karakteristike LED diode pri $\nu = 50\si{Hz}$}
\end{slika}

\section*{Dioda}

Pri navadni diodi imamo napetost:

\[
  U = (0.5 \pm 0.02)\si{V}  
\]

Pri Zenerjevi diodi pa imamo napetost:

\[
  U_{Zener} = (-6.58 \pm 0.01)\si{V}  
\]

\begin{slika}[H]
  \centering
  \includegraphics{dioda}
  \caption{\small Skica karakteristike Zenerjeve diode pri $\nu = 10\si{kHz}$}
\end{slika}

\section*{Baterija in akumulator}

Te elementi imajo iz koordinatnega izhodišča izmaknjeno napetost, saj povzročajo napetost, ne da bi tok tekel skozi njih. V AC načinu lahko natančneje določimo njihovo notranjo upornost.

Za baterijo pri $\nu = 50\si{Hz}$ AC napetosti: 

\[
  U = (8 \pm 0.5)\si{V}  
\]

in 

\[
  R_{int} = (1.5 \pm 0.2)\Omega  
\]

Za akumulator pri $\nu = 50 \si{Hz}$ AC napetosti: 

\[
  U = (1.3 \pm 0.05)\si{V}  
\]

in 

\[
  R_{int} = (1.8 \pm 0.4)\Omega  
\]


\begin{slika}[H]
  \centering
  \includegraphics{baterija}
  \caption{\small Skica karakteristike baterije in akumulatorja pri $\nu = 50\si{Hz}$ AC napetosti.}
\end{slika}


\end{document}